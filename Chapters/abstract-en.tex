%%%%%%%%%%%%%%%%%%%%%%%%%%%%%%%%%%%%%%%%%%%%%%%%%%%%%%%%%%%%%%%%%%%%
%% abstrac-en.tex
%% UNL thesis document file
%%
%% Abstract in English
%%%%%%%%%%%%%%%%%%%%%%%%%%%%%%%%%%%%%%%%%%%%%%%%%%%%%%%%%%%%%%%%%%%%
An increasing number of web applications run totally or partially in the client machines - from collaborative editing tools to multi-user games. Avoiding to continuously contact the server allows to reduce latency between clients and to minimize the load on the centralized component. Novel implementation techniques, such as WebRTC, allows to address network problems that previously prevented these systems to be deployed in practice. Legion is a newly developed framework that exploits these mechanisms, allowing client web applications to replicate data from servers, and synchronize these replicas directly among them.\par
	This work aims to extend the current Legion framework with the integration of an additional legacy storage system that can be used to support web applications, Antidote. We study the best way to support Legion's data model into Antidote, we design a synchronization mechanism, and finally we measure the performance cost of such an integration.
	
% Palavras-chave do resumo em Inglês
\begin{keywords}
distributed storage systems; CRDT; Legion; Antidote.
\end{keywords} 
