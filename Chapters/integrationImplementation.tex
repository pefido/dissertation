\chapter{Integration Implementation}
\label{cha:integration_implementation}

In this chapter we will focus on the implementation of the integration mechanism, describing the work done in each component, the connection between components and emphasizing some important implementation details.\par
	This chapter starts by
\begin{enumerate*}[(i)]

\item describing the supported data types and how they translate between systems, then it will 

\item enumerate the changes we deemed needed to Antidote, which are mainly making the system more language abstract to the exterior. After this, we will describe the 

\item changes made to Legion, in the objects server. After the component's description we will get a more in depth description of the update propagation flow, from

\item Legion to Antidote, and then from

\item Antidote to Legion.

\end{enumerate*}	

\section{Supported Data Types}
\label{sec:supported_data_types}
This section describes the supported data types from Legion and how they translate to the correspondent Antidote data types.

\subsection{Sets}
\label{sec:sets}
As part of this integration, we support synchronizing sets between Legion and Antidote.\par Legion sets are an implementation of the operation-based ORSet CRDT, as described in \cite{crdt}. In the operation-based model, updates are propagated as operations. Antidote also supports the same operation-based propagation model and uses ORSets for supporting sets.\par
	When an operation is done in a system's ORSet, the same operation will be issued to the other system's ORSet. This process is straightforward, but we need to keep track of the unique identifiers that pair with each element of the set. Each system creates a different identifier when an element is stored, so we keep a conversion list for the synchronizing elements. This is important, for example, when deleting an element from the set. We want to delete an element with a particular unique id and not any element.\par
	We also need to guarantee that an element is sent to Antidote only once, because there can be more than one objects server propagating the changes to the same Antidote node. This is done by keeping the propagated operations identifiers in Antidote, so it can be queried when an operation is to be propagated.

\subsection{Counters}
\label{sec:counters}
One of the supported data types in Legion are Counters. They are internally implemented as a state based PN-Counter, as described in \cite{crdt}. An operation includes the full object state. The execution of an operation in a remote site is equivalent to the merge function of CvRDTs. In order to match this data type in Antidote, we used its Counter data type, which is implemented as an operation based PN-Counter.\par
	To synchronize the two counters, we must execute increments and decrements based on who issued the update. Since the counter implementation of each system is different, if the update is issued in Legion, the state of the counter is sent in the update and we need to apply the current change in the counter to the correspondent Antidote counter. To do so, we take the difference between the current value of the Legion object and the value of the counter in the last time the counter was synchronized with Antidote. The difference is the value added/ subtracted from the counter by Legion clients, which should be added to the Antidote counter.\par
	If the update is issued in Antidote, the update will contain the increment or decrement. This delta is added or subtracted to the Legion counter value and stored.\par
	Again, we have to keep a list of propagated operations in Antidote in order to avoid duplicated operations. We also need to keep with the Legion counter, the information of the last Antidote version a given Legion replica synchronized with.

\section{Antidote changes}
\label{sec:antidote_changes}
In this section we will describe the changes to Antidote that were needed in order to make the integration process possible.\par
	Antidote is a storage system based in Riak Core and it is still in a fresh stage of development. It is written in Erlang and it offers two interfaces for supporting clients: 
\begin{enumerate*}[(i)]
\item a distributed Erlang interface and 
\item a protocol buffer interface.
\end{enumerate*}
\par
	The first one uses distributed Erlang and can only be used by an Erlang client, which is of no use to us in our work.\par
	The second one uses Google's protocol buffer serialization mechanism, as detailed in section \ref{sec:data_serialization}, to offer an interface that can be language independent. This would suit our needs since we want to be able to communicate with Antidote from a Legion objects server written in JavaScript. In order to use this interface, some changes had to be made. Some protocol buffer message values were raw Erlang binaries, which are extremely difficult to encode/ decode in JavaScript. So, we need to change it to use a language independent format. Also, some important methods that are available in the distributed Erlang interface need to be included in the protocol buffer API, such as the operations that read objects in its raw state, including internal metadata instead of returning only the value of the objects. In summary, the following functionalities had to be supported by Antidote:
	\begin{itemize}
	\item The protocol buffer messages can only contain basic data types, like numbers or strings for the interface to be language independent.
	\item API methods have to return the object values but also internal metadata, such as commit timestamps and unique identifiers from operations and data structure elements like sets.
	\item Antidote has to expose an API method that returns the contents of the operation log for an object since a certain timestamp.
	\end{itemize}
\par
	With the help from Antidote's development team, these functionalities were implemented. The missing interface methods were already implemented internally, so it was only required to make an interface method to export those.\par
	In order to make the interface language independent, it uses a JSON data structure over protocol buffer messages. This was possible by adding a JSON encoding/ decoding component in Antidote's protocol buffer interface. Although this might not be the best solution, since it partially defeats the purpose of protocol buffers, it was the more practical solution regarding this thesis time span. The result of this new interface is a single protocol buffer method that encapsulates every JSON message sent as a request to Antidote. This JSON request is then parsed internally in Antidote.\par
	Summing up, in listing \ref{lst:proto1} we show the protocol buffer interface methods before this change, and listing \ref{lst:proto2} shows the only method that these were encapsulated into, where the 'value' field is a JSON object with the request.\par
	These changes enabled us to proceed with the system integration. Algorithm \ref{antidote_interaction} allow us to understand the flow that Antidote takes after receiving a request by showing how Antidote handles the incoming requests from both clients and Legion nodes. When handling a write request, Antidote propagates the update internally and replies with information from that transaction. When the request is a read, Antidote replies with the object itself, including unique identifiers and transaction information. The requests can include a timestamp that represents the system snapshot on which the read/ write will be executed.
	
\begin{algorithm}[H]
\caption{Antidote request interaction}\label{antidote_interaction}
\begin{algorithmic}[1]
\algblock{Upon}{End}
\Upon { $request$:}
  \algblock{Switch}{End}
  \Switch { $request.type$}
    \algblock{case}{End}
    \case { "beginTransaction"}
      \State {$txId$ $\gets$ beginTransaction($request.timestamp$)}
      \State {Send reply($txID$)}
    \End
    \case { "commitTransaction"}
      \State {$(txInfo, content)$ $\gets$ commitTransaction($request.txId$)}
      \State {Send reply($txInfo, content$)}
    \End
    \case { "write"}
      \State {$answer$ $\gets$ updateObjects($request.objectId$)}
      \If {answer}
        \State {Send reply(OK)}
      \EndIf
    \End
    \case { "read"}
      \State {$result$ $\gets$ getObjects($request.objectId$)}
       \State {Send reply($result$)}
    \End
    \case { "getLogOperations"}
      \State {$logOps$ $\gets$ getLogOperations($request.timestamp$)}
      \State {Send reply($logOps$)}
    \End
  \End
\End
\end{algorithmic}
\end{algorithm}

\begin{lstlisting}[caption={Protocol Buffer interface methods before},label={lst:proto1}]
message ApbReadObjects {
  repeated ApbBoundObject boundobjects = 1;
  required bytes transaction_descriptor = 2;
}

message ApbGetObjects {
  repeated ApbBoundObject boundobjects = 2;
}

message ApbUpdateObjects {
  repeated ApbUpdateOp updates = 1;
  required bytes transaction_descriptor = 2;
}

message ApbStartTransaction {
  required ApbVectorclock timestamp = 1;
  optional ApbTxnProperties properties = 2;
}

message ApbCommitTransaction {
  required bytes transaction_descriptor = 1;
}

message ApbGetLogOperations {
  repeated ApbVectorclock timestamps = 2;
  repeated ApbBoundObject boundobjects = 3;
}
\end{lstlisting}

\begin{lstlisting}[caption={Protocol Buffer interface methods after},label={lst:proto2}]
message ApbJsonRequest {
  required bytes value = 1;
}
\end{lstlisting}

\section{Legion changes}
\label{sec:legion_changes}
The majority of the code from the implementation was developed in Legion, more specifically in the objects server component. This is where the logic of the synchronization resides.\par
	To manage the synchronization between Legion and Antidote there two main parts that needed to be implemented: 
\begin{enumerate*}[(i)]
\item propagate updates that reach the objects server from a legion node, and 
\item periodically check for updates that were issued directly to Antidote.
\end{enumerate*}

\subsection{Legion to Antidote flow}
\label{sec:legion_changes_legion_to_antidote_flow}
As mentioned before in secton \ref{sec:legion_to_antidote_flow}, every update issued by a Legion node will be processed in the objects server. Algorithm \ref{objects_server_update_propagation} shows the steps needed for the objects server to propagate each update to Antidote. A message that reaches the objects server can be of several types, we want to parse the ones having the type 'contentFromNetwork'. By catching this event we know that an update has been issued, so now we can parse the message and get the details that we need to propagate the operation to Antidote. Listing \ref{lst:update_example} shows an example of an incoming update message.\par
	After parsing the message a request to Antidote checks if that operation was already propagated. In case it was not, two separate requests are sent to Antidote, one for the actual data, and the other for the meta-data. To write both in Antidote, the protocol buffer interface is used by calling the "updateObjects" method with the element and operation. Both requests are made under mutual exclusion. For this, we rely on the ZooKeeper service. Legion's objects server needs to obtain a lock before the update is done, releasing it after. ZooKeeper acts as a centralized coordination service, by granting locks and maintaining the locking state per object. The meta-data held in Antidote represents the operations that were already propagated from Legion to Antidote, as well as CRDT specific unique identifiers.
	
\begin{lstlisting}[caption={Legion update content message example},label={lst:update_example}]
{
  type:"OS:C",
  sender:1,
  ID:7012120591,
  content:{
    type:"OPLIST",
    objectID:"objectID2",
    operations:[
      {
        dependencyVV:{},
        opID:1,
        result:{
          element:"a",
          unique:"6651411189259640"
        },
        opName:"add",
        clientID:"1"
      }
    ]
  },
  destination:"localhost:8004"
}
\end{lstlisting}


\begin{algorithm}
\caption{Objects Server Legion to Antidote update propagation}\label{objects_server_update_propagation}
\begin{algorithmic}[1]
\algblock{Upon}{End}
\Upon { message:}
  \If {$message.type$ = "contentFromNetwork"}
      \State {$opInfo$ $\gets$ parse($message.content$)}
      \State {$doneOps$ $\gets$ antidote.getObjects("DONE\_OPS")}
      \If {!($opInfo.id$ in $doneOps$)}
      	\State {zooKeeper.lock($opInfo.objectId$)}
      	\State {$txId$ $\gets$ antidote.beginTx($lastSeenTimestamp$)}
        \State {antidote.updateObjects("DONE\_OPS", add, $opInfo.id$)}
        %\State {zooKeeper.unlock($opInfo.objectId$)}
        \State {antidote.updateObjects($opInfo.objectID$, $opInfo.op$)}
        \State {antidote.commit($txId$)}
        \Upon { reply($txInfo$):}
          \State {$txId$ $\gets$ antidote.beginTx($lastSeenTimestamp$)}
          \If {$opInfo.CRDTType$ = 'set'}
            \State {$metadata$ $\gets$ parse($opInfo$, $txInfo$)}
            \State {antidote.updateObjects($metadata.objectID$, $metadata.op$, $metadata.tokens$)}
          \ElsIf {$opInfo.CRDTType$ = 'counter'}
            \State {$metadata$ $\gets$ parse($opInfo$, $txInfo$)}
            \State {antidote.updateObjects($metadata.objectID$, $metadata.op$, $metadata.opId$)}
          \EndIf
          \State {antidote.commit($txId$)}
          \State {zooKeeper.unlock($opInfo.objectId$)}
        \End
      \EndIf
  \EndIf
\End
\end{algorithmic}
\end{algorithm}

\subsection{Antidote to Legion flow}
\label{sec:legion_changes_antidote to legion flow}
As mentioned in section \ref{sec:antidote_to_legion_flow}, the process of propagating updates from Antidote to the Legion network is based on periodic polling.\par
	Algorithm \ref{objects_server_update_propagation2} shows the pseudo-code of the objects server work in propagating updates from Antidote to Legion. In order to periodically check for new updates in Antidote, there is a method that starts running when a Legion's objects server starts. This method will fire an event every two seconds and fetches Antidote's operations log. This request is done via the protocol buffer interface and it contains the object identifiers to search for, as well as a given system timestamp that represents the last seen snapshot of Antidote by that Legion objects server. The reply will be a list of executed operations since the given timestamp. An example of the operations log response is shown in listing \ref{lst:log_example}.\par
	If there exists new updates to those objects, we first check if each update was already propagated to Legion. If not, these will be locally executed in the objects server, and then propagated to the all the Legion nodes in the group. Another request is made to Antidote, in order to keep the track of the unique identifiers from Antidote to Legion. This includes updating the correspondence between operation identifiers and CRDT specific tokens.\par
	At the end of this process we also need to update the last seen timestamp of Antidote's state in the objects server. This timestamp corresponds to the commit time of the last update in the operations log. These operations are performed in mutual exclusion per object, using the ZooKeeper service.
	
\begin{algorithm}[H]
\caption{Objects Server Antidote to Legion update propagation}\label{objects_server_update_propagation2}
\begin{algorithmic}[1]
\algblock{Every}{End}
\Every { 2 seconds:}
      \State {$opList$ $\gets$ antidote.getLogOperations($elements$, $lastSeenTimestamp$)}
      \ForEach {$op \in opList$}
        \State {$doneOps$ $\gets$ antdote.getObjects($doneOps$)}
        \If {!($op.id$ in $doneOps$)}
          \State {zooKeeper.lock($op.objectId$)}
          %\State {$opInfo$ $\gets$ apply($op$)}
          
          \If {antidote.hasKey("OP\_MAP", $op.id$) }
            \State {$(antId, legId)$ $\gets$ antidote.getObjects("OP\_MAP", $op.id$)}
            \State {$update$ $\gets$ convertToLegion($op$, $legId$)}
          \Else {}
            \State {$update$ $\gets$ convertToLegion($op$)}
            \State {antidote.updateObjects("OP\_MAP", $op.id$, $update.id$) }
          \EndIf          
          
          \If {$op.CRDTType$ = 'set'}
            \State {$metadata$ $\gets$ parse($op$, $txInfo$)}
            \State {antidote.updateObjects($metadata.objectID$, $metadata.op$, $metadata.tokens$)}
          
          
          \EndIf
          \State {propagate($op$)}
          \State {$lastSeenTimestamp \gets opList.last.timestamp$}
          \State {zooKeeper.unlock($op.objectId$)}
        \EndIf
      \EndFor
\End
\end{algorithmic}
\end{algorithm}
	
\begin{lstlisting}[caption={Antidote log response example},label={lst:log_example}]
{log_operations:[
  {
    clocksi_payload:{
        key: {json_value:"objectID2"}
    },
    type:"crdt_orset",
    update:{
      add:[
        {json_value:"b"},
        {binary64:"YySEUYb9uvr4EMO3glI8BWTKurw="}
      ]
    },
       commit_time:[
         {dcid:["antidote@127.0.0.1",1472,53665,947954]},
         1472053805871290
   ]
  }
]}
\end{lstlisting}


%\section{Operation Propagation}
%\label{sec:operation_propagation}

%\section{Legion to Antidote flow}
%\label{sec:legion_to_antidote_flow}

%\section{Antidote to Legion flow}
%\label{sec:antidote to legion flow}