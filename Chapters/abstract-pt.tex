%%%%%%%%%%%%%%%%%%%%%%%%%%%%%%%%%%%%%%%%%%%%%%%%%%%%%%%%%%%%%%%%%%%%
%% abstrac-pt.tex
%% UNL thesis document file
%%
%% Abstract in Portuguese
%%%%%%%%%%%%%%%%%%%%%%%%%%%%%%%%%%%%%%%%%%%%%%%%%%%%%%%%%%%%%%%%%%%%
Um crescente número de aplicações \textit{web} é executado totalmente ou parcialmente nas máquinas do cliente - desde ferramentas de edição colaborativa até jogos de vários utilizadores. Ao cortar a execução de operações no servidor é possível reduzir a latência entre clientes e minimizar a carga no servidor. Novas tecnologias, como WebRTC, permitem atacar problemas que anteriormente afectavam a implementação destes sistemas num ambiente de produção. O Legion, é uma \textit{framework} recém desenvolvida que explora estes mecanismos, permitindo que aplicações \textit{web} do lado do cliente consigam fazer replicação de dados e sincronização directa entre elas.\par
	Neste trabalho pretende-se estender a \textit{framework} Legion através da integração com um sistema de armazenamento já existente que pode ser utilizado para suportar aplicações \textit{Web}, o Antidote. Estudamos o melhor método para suportar o modelo de dados do Legion no Antidote, desenhamos o mecanismo de sincronização, e finalmente medimos o custo no desempenho de tal integração.

% Palavras-chave do resumo em Português
\begin{keywords}
sistemas de armazenamento distribuido; CRDT; Legion; Antidote.
\end{keywords}
% to add an extra black line
