\chapter{Integration Design and Architecture}
\label{cha:integration_design_and_architecture}

In this chapter it will be presented the path taken in the integration design process, as well as the architecture on which the implementation has its basis.\par
	The main topics for this section will be displayed as it follows:
\begin{itemize}
\item Technical decisions

\item Components Overview

\item Legion to Antidote flow

\item Antidote to Legion flow
\end{itemize}

\section{Technical decisions}
\label{sec:technical_decisions}
In the starting point of this thesis, the main objective was to integrate Legion with two data storage systems that also used CRDTs, Riak and Antidote. Riak is more enterprise oriented, with several real world use cases, while Antidote is still very fresh and under development. Even thought Antidote uses Riak Core, the two systems diverge a lot based on design options and it's objectives. We started by speaking to a few developers from each of the systems, being able to conclude that Antidote's development environment would be much more beneficial for us regarding support. This was a critical matter since no one directly involved in this thesis was an expert in either of the systems, neither we had much experience in the programming language used (Erlang). With this in mind, we followed to start the integration with Antidote.\par
Our first integration design thought was to use the CRDT's data structures of both systems, translate one to another, and then merge them in order to propagate the state of the system. Once again we talked to the development team as to get a guideline of how this could be done. As we understood, it was not the intended design path to expose the internal CRDT's data structure to the outside of the system. The system's CRDTs were meant to only be accessed as an object abstraction.\par
After acknowledging this, the design of the integration was changed to be based on propagating operations, by executing on one system the operation that was made on the other.


\section{Architecture Overview}
\label{sec:architecture_overview}
Integrating Legion with other storage systems, should be made in a modular and decoupled way, so if one decides to add support for another storage system, the recipe should be similar. This can be achieved by having system components that can easily be replaced.\par
	The integration of Legion and Antidote can be depicted into the following components:
\begin{description}

\item[Antidote] \hfill \\
The storage system to be integrated with Legion. Antidote will act as a persistent, stable, client-server model storage system. It will be synchronized with Legion, so both will have the entire set of updated objects in the system. An Antidote node can have it's data updated directly from an antidote client, or from an update that was propagated from Legion nodes.\par
Besides that, Antidote will also keep meta-data related to the system synchronization, such as propagated operations identifiers and elements' unique tokens.

\item[Antidote Client] \hfill \\
This component is one of the entry points of information into the system. An Antidote client can manipulate the system's data set by making calls to modify the state of Antidote's objects.\par
	This client is written in javascript and it uses the integration proxy in order to make read and write requests.


\item[Legion Client] \hfill \\
This component is a Legion instance in the system. It acts as an entry point of information to the system and it also as the responsibility to keep it self and the other Legion nodes updated and synchronized.


\item[Legion's Objects Server] \hfill \\
This component keeps a global view of what is happening in the Legion network. There can be one or more objects servers in the system, and Legion's synchronization events will pass through at least one of these. In the objects server also resides the logic for the synchronization between Antidote and Legion. This is done by watching for object update events and then propagating them.\par
This component is written in NodeJS and it uses the integration proxy for the communication with Antidote, by making read and write calls.

\item[Integration Proxy] \hfill \\
This component is needed to establish communication with Antidote. It takes advantage of Antidote's protocol buffer interface to make read and write calls.\par
The proxy can receive requests from either a Antidote client or a Legion's object server synchronization method. The request is parsed and a call is sent to Antidote. When the answer from antidote is received, it is redirected to who made the request. This component is written in NodeJS.

\end{description}

\section{Legion to Antidote flow}
\label{sec:legion_to_antidote_flow}

\section{Antidote to Legion flow}
\label{sec:antidote to legion flow}