%%%%%%%%%%%%%%%%%%%%%%%%%%%%%%%%%%%%%%%%%%%%%%%%%%%%%%%%%%%%%%%%%%%%
%% abstrac-pt.tex
%% UNL thesis document file
%%
%% Abstract in Portuguese
%%%%%%%%%%%%%%%%%%%%%%%%%%%%%%%%%%%%%%%%%%%%%%%%%%%%%%%%%%%%%%%%%%%%
Um crescente número de aplicações \textit{web} é executado totalmente ou parcialmente nas máquinas do cliente - desde ferramentas de edição colaborativa até jogos de vários utilizadores. Ao cortar a execução de operações no servidor é possível reduzir a latência entre clientes e minimizar a carga no servidor. Novas tecnologias, como WebRTC, permitem atacar problemas que anteriormente afectavam a implementação destes sistemas num ambiente de produção. O Legion, é uma \textit{framework} recém desenvolvida que explora este mecanismos, permitindo que aplicações \textit{web} do lado do cliente consigam fazer replicação de dados e sincronização directa entre elas.\par
	Neste trabalho pretende-se estender a \textit{framework} Legion de duas formas significativas: \begin{enumerate*}[(i)]
	\item integração com sistemas de armazenamento já existentes que são tipicamente usados para suportar aplicações \textit{Web}. Também pretendemos estudar o impacto que diferentes modelos de dados usados por estes sistemas de armazenamento assim como as suas interfaces têm no desempenho do sistema resultante da integração do Legion com estes sistemas;
	\item enriquecer o Legion para suportar consistência causal alavancando nos mecanismos disponibilizados pelos serviços de armazenamento centralizados.
\end{enumerate*}

% Palavras-chave do resumo em Português
\begin{keywords}
sistemas de storage distribuido; CRDT; Legion; Antidote.
\end{keywords}
% to add an extra black line
